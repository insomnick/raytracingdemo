% Eine neue Farbe für keywords im Code und für links
\definecolor{codeblue}{HTML}{4178f2}

% Entfernt den roten Rahmen um klickbare Links und macht Links blau
\hypersetup{hidelinks, colorlinks=true, allcolors=codeblue}

\setkomafont{publishers}{\large}
\setkomafont{date}{\large}
\setkomafont{author}{\Large}

% Anpassen des Zitierstils
\DeclareLabelalphaTemplate{
  \labelelement{
    \field[strwidth=1,strside=left]{labelname}
  }
  \labelelement{
    \field[strwidth=1,strside=right]{labelname}
  }
  \labelelement{
    \field[strwidth=1]{year}
  }
}

% Anpassung der Ausgabe der Autoren
\renewcommand*{\labelalphaothers}{}
\DeclareNameAlias{sortname}{last-first}

% Einstellung für das listings-Paket zum Setzen von Code
% Generelle Einstellungen
\lstset{%
  frame=single,
  framesep=0mm,
  framexleftmargin=7mm,
  xleftmargin=8mm,
  framerule=1.5pt,
  rulecolor=\color{black},
  backgroundcolor=\color{white},
  basicstyle=\ttfamily,
  keywordstyle=\bfseries\color{codeblue},
  commentstyle=\color{gray},
  numbers=left,
  stepnumber=1,
  numbersep=1mm,
  numberstyle=\sffamily\color{gray!80!black}\footnotesize,
  numberblanklines=true,
  escapeinside={\%*}{*)},
  inputencoding=utf8
}

% Definiere Sprache 'pseudocode' für Codelistings
\lstdefinelanguage{pseudocode}{
    keywords={
        function,
        if,
        then,
        else,
        for,
        while,
        do,
        in,
        True,
        False,
        Array,
        set,
        repeat,
        break,
        continue,
        Eingabe,
        Ausgabe,
        return,
        print,
        exit
    },
    sensitive=false,
    delim=[l][keywordstyle]{:},
    comment=[l][commentstyle]{\#}
}

% Dieser von Stackoverflow kopierte Block sorgt dafür, dass Umlaute in
% Code-Listings funktionieren.
% https://tex.stackexchange.com/questions/24528/having-problems-with-listings-and-utf-8-can-it-be-fixed
\lstset{
  inputencoding = utf8,  % Input encoding
  extendedchars = true,  % Extended ASCII
  literate      =        % Support additional characters
    {á}{{\'a}}1  {é}{{\'e}}1  {í}{{\'i}}1 {ó}{{\'o}}1  {ú}{{\'u}}1
    {Á}{{\'A}}1  {É}{{\'E}}1  {Í}{{\'I}}1 {Ó}{{\'O}}1  {Ú}{{\'U}}1
    {à}{{\`a}}1  {è}{{\`e}}1  {ì}{{\`i}}1 {ò}{{\`o}}1  {ù}{{\`u}}1
    {À}{{\`A}}1  {È}{{\'E}}1  {Ì}{{\`I}}1 {Ò}{{\`O}}1  {Ù}{{\`U}}1
    {ä}{{\"a}}1  {ë}{{\"e}}1  {ï}{{\"i}}1 {ö}{{\"o}}1  {ü}{{\"u}}1
    {Ä}{{\"A}}1  {Ë}{{\"E}}1  {Ï}{{\"I}}1 {Ö}{{\"O}}1  {Ü}{{\"U}}1
    {â}{{\^a}}1  {ê}{{\^e}}1  {î}{{\^i}}1 {ô}{{\^o}}1  {û}{{\^u}}1
    {Â}{{\^A}}1  {Ê}{{\^E}}1  {Î}{{\^I}}1 {Ô}{{\^O}}1  {Û}{{\^U}}1
    {œ}{{\oe}}1  {Œ}{{\OE}}1  {æ}{{\ae}}1 {Æ}{{\AE}}1  {ß}{{\ss}}1
    {ç}{{\c c}}1 {Ç}{{\c C}}1 {ø}{{\o}}1  {å}{{\r a}}1 {Å}{{\r A}}1
    {ã}{{\~a}}1  {õ}{{\~o}}1  {Ã}{{\~A}}1 {Õ}{{\~O}}1
    {ñ}{{\~n}}1  {Ñ}{{\~N}}1  {¿}{{?`}}1  {¡}{{!`}}1
    {°}{{\textdegree}}1 {º}{{\textordmasculine}}1 {ª}{{\textordfeminine}}1
}

% Vordefinierte Umgebungen für Theoreme/Lemmas/Definitionen
\theoremstyle{plain} % Alle newtheorems die hier nach kommen bekommen den 'plain' Stil
\newtheorem{theorem}{Theorem}
\newtheorem{lemma}[theorem]{Lemma}
\newtheorem{definition}[theorem]{Definition}

% Stelle Theorem/Lemma/Definitionsnummerierung die Abschnittsnummer voran
\numberwithin{theorem}{section}
